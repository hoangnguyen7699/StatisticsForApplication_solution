\documentclass[10pt]{article}
 
\usepackage[margin=1in]{geometry} 
\usepackage{bbm}
\usepackage{amsmath,amsthm,amssymb, graphicx, multicol, array}
 
\newcommand{\N}{\mathbb{N}}
\newcommand{\Z}{\mathbb{Z}}
 
\newenvironment{problem}[2][Problem]{\begin{trivlist}
\item[\hskip \labelsep {\bfseries #1}\hskip \labelsep {\bfseries #2.}]}{\end{trivlist}}



\begin{document}
\title{Problem Set 2}
\author{Hoang Nguyen, Huy Nguyen}
\maketitle
    
\begin{problem}{1} 
QQ-plots \\
\\
QQ-Plot 1: Laplace distribution with parameter $\sqrt{2}$ (heavier-heavier) \\
QQ-Plot 2: Uniform distribution on [-$\sqrt{3}$, $\sqrt{3}$] (lighter-lighter) \\
QQ-Plot 3: standard Gaussian distribution \\
QQ-Plot 4: exponential distribution with parameter 1 (lighter-heavier)\\
QQ-Plot 5: e Cauchy distribution (heavier-heavier)\\







\end{problem}

\begin{problem}{2}
\item 1. Mark as update

\item 2. \\
Let A(t) be the cdf for $U_i's$ \\
We have :\\
\[A(t)= \mathbb{P}(U_i \leq t)=\mathbb{P}\big(F(X_i)\leq t\big)\]
Because F is increasing function:
\[A(t)= \mathbb{P}\big(X_i\leq F^{-1}(t)\big)=F\big(F^{-1}(t) \big)=t\]
Notice that $0\leq t \leq 1$ \\
Hence, distributions of the $U_i's$ is Uniform(0,1)\\
Do similarly we get distributions of the $V_i's$ is also Uniform(0,1) \\

\item 3.\\
a) \\
\[T_{n,m}=\sup_{t\in \mathbb{R}}\big|F_n(t)-G_m(t) \big|=\sup_{t\in \mathbb{R}}\big|\frac{1}{n} \sum_{i=1}^{n}\mathbbm{1}(X_i < t) -\frac{1}{m} \sum_{i=1}^{m}\mathbbm{1}(Y_i < t) \big|\]
b) \\
\[T_{n,m}=\sup_{t\in \mathbb{R}}\big|F_n(t)-G_m(t) \big|=\sup_{t\in \mathbb{R}}\big|\frac{1}{n} \sum_{i=1}^{n}\mathbbm{1}(X_i < t) -\frac{1}{m} \sum_{i=1}^{m}\mathbbm{1}(Y_i < t) \big|\]

F, G are increasing function 
\[\Rightarrow T_{n,m}= \sup_{t\in \mathbb{R}}\big|\frac{1}{n} \sum_{i=1}^{n}\mathbbm{1}\big(F(X_i) < F(t)\big) -\frac{1}{m} \sum_{i=1}^{m}\mathbbm{1}\big(G(Y_i) < G(t)\big) \big|\]
\[=\sup_{t\in \mathbb{R}}\big|\frac{1}{n} \sum_{i=1}^{n}\mathbbm{1}\big(U_i < F(t)\big) -\frac{1}{m} \sum_{i=1}^{m}\mathbbm{1}\big(V_i < G(t)\big) \big|\]
$H_0$ is true $\Rightarrow$ F(t)=G(t)=x ($0 \leq x \leq 1$):
\[T_{n,m}=\sup_{0 \leq x \leq 1}\big|\frac{1}{n} \sum_{i=1}^{n}\mathbbm{1}\big(U_i < x\big) -\frac{1}{m} \sum_{i=1}^{m}\mathbbm{1}\big(V_i < x\big) \big|\]
c)\\If H0 is true, the joint distribution of the n + m random variables $U_1,....,U_n,V_1,...,V_m$ is the joint distribution of the n + m Uniform(0,1)\\
d) \\
\[T_{n,m}=\sup_{0 \leq x \leq 1}\big|\frac{1}{n} \sum_{i=1}^{n}\mathbbm{1}\big(U_i < x\big) -\frac{1}{m} \sum_{i=1}^{m}\mathbbm{1}\big(V_i < x\big) \big|\]
\[=\sup_{0 \leq x \leq 1}\big|\frac{1}{n} \sum_{i=1}^{n}Ber\big(\mathbb{P}(U_i < x)\big) -\frac{1}{m} \sum_{i=1}^{m}Ber\big(\mathbb{P}(V_i < x)\big) \big|\]
\[=\sup_{0 \leq x \leq 1}\big|\frac{1}{n} \sum_{i=1}^{n}Ber\big(x\big) -\frac{1}{m} \sum_{i=1}^{m}Ber\big(x\big) \big|\]

Hence, $T_{n,m}$ is pivotal\\
f) By CLT, we have
\[\sqrt{n}\big(F_n(t)-F(t)\big) \xrightarrow[n \mapsto\infty ]{(d)}  \mathbb{N}\big(0, F(t)(1-F(t)) \big) \]
\[\Leftrightarrow F_n(t)-F(t)\xrightarrow[n \mapsto\infty ]{(d)} \mathbb{N}\big(0,\frac{F(t)(1-F(t))}{n}  \big) \]\\

Similarly, we have:
\[Q_m(t)-Q(t)\xrightarrow[m \mapsto\infty ]{(d)} \mathbb{N}\big(0,\frac{Q(t)(1-Q(t))}{m}  \big) \]\\

From this, if $H_0$ is true ($F(t)=Q(t)$),  by subtracting 2 equations above, we have:
\[F_n(t)-Q_m(t) \xrightarrow[n,m \mapsto\infty ]{(d)} \mathbb{N}\big(0, \frac{F(t)(1-F(t))}{n} +\frac{Q(t)(1-Q(t))}{m}  \big)\]\\

$H_0 $ is true:
\[\Leftrightarrow  F_n(t)-Q_m(t)  \xrightarrow[n,m \mapsto\infty ]{(d)} \mathbb{N}\big(0,\sqrt{\frac{m+n}{mn}}F(t)(1-F(t)) \big)\]
\[\Leftrightarrow \sqrt{\frac{mn}{m+n}}\big( F_n(t)-Q_m(t) \big)  \xrightarrow[n,m \mapsto\infty ]{(d)} \mathbb{N}\big(0,F(t)(1-F(t)) \big)\]
\[\Rightarrow\sqrt{\frac{mn}{m+n}} \sup_{t\in \mathbb{R}}\big| F_n(t)-Q_m(t) \big|  \xrightarrow[n,m \mapsto\infty ]{(d)} \sup_{t\in \mathbb{R}} \big|\mathbb{B}(F(t))\big| \]

where $\mathbb{B}$ is a Brownian bridge on [0,1]\\

Define a test $T_n=\sqrt{\frac{mn}{m+n}} \sup_{t\in \mathbb{R}}\big| F_n(t)-Q_m(t) \big| $ , by Donsker's theorem, if $H_0$ is true, then $T_n \xrightarrow[n,m \mapsto\infty ]{(d)} Z$ where X has a known distribution


\end{problem}

\begin{problem}{3}
\item 1.
Mark as update
\item 2.
We have: $R_i$, $R_j$ take the values 1,2,..n\\

With a,b=1,2,..n :
\[\mathbb{P}(R_i=a, R_j=b)= (n-2)!\big(\frac{n-2}{n} \big)^{n-2} \] 
\[\mathbb{P}(R_i=a)= \frac{1}{n}\]
\[\mathbb{P}(R_j=b)= \frac{1}{n}\]
\[\Rightarrow \mathbb{P}(R_i=a, R_j=b)\neq \mathbb{P}(R_i=a)\mathbb{P}(R_j=b)  \]

Hence, $R_1,...R_{n}$ are not independent


\item 3.
We have: \\

$R_i$ takes the values in (1,2,..,n)
\[\Rightarrow \mathbb{P}(R_i=a)=\begin{cases} \frac{1}{n} & \mbox{if } a\mbox{ =1,2,..,n} \\ 0, &  \mbox{Otherwise } \end{cases}\]
\[\Rightarrow R  \backsim Uniform(1, n)\]

Similarly:
\[\Rightarrow Q  \backsim Uniform(1, n)\]

Hence, R, Q do not depend on  distribution of $X_i$'s, $Y_i$'s

\item 4.
If $H_0$ is true:
\[\mathbb{P}(R, Q)=\frac{1}{n^2}\]
\[\mathbb{P}(R)=\frac{1}{n}\]
\[\mathbb{P}(Q)=\frac{1}{n}\]
\[\Rightarrow \mathbb{P}(R, Q)=\mathbb{P}(R)\mathbb{P}(Q)\]

Hence, ($R_1$, $R_2$,...,$R_n$) and ($Q_1$, $Q_2$,...,$Q_n$) are independent

\item 5.
If $H_0$ is true, ($R_1$, $R_2$,...,$R_n$, $Q_1$, $Q_2$,...,$Q_n$) are the iid Uniform(1,n) distributon $\Rightarrow$ joint distribution of them does not depend on the distribution of the
original sample

\item 6.
\[T_n=\frac{\sum_{i=1}^{n}(R_i-\bar{R_n})(Q_i-\bar{Q_n})}{\sqrt{\sum_{i=1}^{n}(R_i-\bar{R_n})^2\sum_{i=1}^{n}(Q_i-\bar{Q_n})^2}}\]
\[T_n=\frac{\sum_{i=1}^n R_iQ_i -\sum_{i=1}^n \bar{R_n}(R_i+Q_i)+\sum_{i=1}^n \bar{R_i}\bar{Q_i}}{\sqrt{\sum_{i=1}^{n}(R_i-\bar{R_n})^2\sum_{i=1}^{n}(Q_i-\bar{Q_n})^2}}\]
\[T_n=\frac{12}{n(n^2-1)}\sum_{i=1}^{n} R_iQ_i- \frac{3(n+1)}{n-1}\]

\item 7.
From question 3 we have R, Q do not depend on  distribution of $X_i$'s, $Y_i$'s\\

$\Rightarrow$ $R_i '$ and $Q_i '$ are also Uniform(1, n)\\

$\Rightarrow S_n$ is the same distribution as $T_n$




\end{problem}

\end{document}