\documentclass[10pt]{article}
 
\usepackage[margin=1in]{geometry} 
\usepackage{amsmath,amsthm,amssymb, graphicx, multicol, array}
 
\newcommand{\N}{\mathbb{N}}
\newcommand{\Z}{\mathbb{Z}}
 
\newenvironment{problem}[2][Problem]{\begin{trivlist}
\item[\hskip \labelsep {\bfseries #1}\hskip \labelsep {\bfseries #2.}]}{\end{trivlist}}



\begin{document}
\title{Problem Set 5}
\author{Hoang Nguyen, Huy Nguyen}
\maketitle
    
\begin{problem}{1}
\item 1.
From exercise 3 in Problem 2, If $Z \thicksim N(0,1)$, we get: $\mathbb{P}\Big[|\frac{\sqrt{n}(\bar{X_i}-\mu)}{\sigma}| \leqslant t\Big]=2\mathbb{P}[Z \leqslant t]-1$.\\
$X_1, X_2,...,X_n$ follow Poisson distribution. Hence, $\mathbb{E}[X_i]=\lambda$ and $\mathbb{V}[X_i]=\lambda$\\
From CTL: $\frac{\sqrt{n}(\bar{X_n}-\lambda)}{\sqrt{\lambda}} \thicksim N(0,1)$. Hence, $\mathbb{P}\Big [ |\frac{\sqrt{n}(\bar{X_n}-\lambda)}{\sqrt{\lambda}}| \leqslant t \Big]=2\mathbb{P}\Big [ \frac{\sqrt{n}(\bar{X_n}-\lambda)}{\sqrt{\lambda}} \leqslant t \Big ]-1=1-\alpha.$\\
Hence, $\mathbb{P}\Big [ \frac{\sqrt{n}(\bar{X_n}-\lambda)}{\sqrt{\lambda}} \leqslant t \Big ]=1- \frac{\alpha}{2}$. Therefore, $t=\phi^{-1}(1- \frac{\alpha}{2}).$\\
We have $ = \mathbb{P}\Big [ |\frac{\sqrt{n}(\bar{X_n}-\lambda)}{\sqrt{\lambda}}| \leqslant \phi^{-1}(1- \frac{\alpha}{2}) \Big ]\rightarrow (1-\alpha)$ as n tends to infinity. This equivalent to:
\[  \mathbb{P}\Big[-\frac{\sqrt{n}(\bar{X_n}-\lambda)}{\sqrt{\lambda}} \leqslant \phi^{-1}(1- \frac{\alpha}{2}) \leqslant \frac{\sqrt{n}(\bar{X_n}-\lambda)}{\sqrt{\lambda}}\Big] \rightarrow 1-\alpha\] 
\[ \mathbb{P}\Big[\bar{X_n}- \frac{\phi^{-1}(1- \frac{\alpha}{2})\sqrt{\lambda}}{\sqrt{n}} \leqslant \lambda \leqslant \bar{X_n}+ \frac{\phi^{-1}(1- \frac{\alpha}{2})\sqrt{\lambda}}{\sqrt{n}}  \Big] =1- \alpha\]
We know: $\bar{X_n} \xrightarrow{P} \mathbb{E}[\bar{X_n}]=\lambda$\\
Hence, 
\[ \mathbb{P}\Big[\bar{X_n}- \frac{\phi^{-1}(1- \frac{\alpha}{2})\sqrt{\bar{X_n}}}{\sqrt{n}} \leqslant \lambda \leqslant \bar{X_n}+ \frac{\phi^{-1}(1- \frac{\alpha}{2})\sqrt{\bar{X_i}}}{\sqrt{n}}  \Big] \geqslant 1- \alpha\]
$\Rightarrow$ L=[$\bar{X_n}- \frac{\phi^{-1}(1- \frac{\alpha}{2})\sqrt{\bar{X_i}}}{\sqrt{n}}, \bar{X_n}+ \frac{\phi^{-1}(1- \frac{\alpha}{2})\sqrt{\bar{X_i}}}{\sqrt{n}}$]


\item 2. 












\item 3. We actually complete this exercise in previous solution.\\
bias$\Big(\bar{X_i}(1-\bar{X_i})\Big)= \mathbb{E}[\bar{X_i}(1-\bar{X_i})] - p(1-p)= \frac{n(n-1)}{n^2} p(1-p) -p(1-p)$.
\item 4. To find an unbiased estimator, we have to find x such that $\frac{xn(n-1)}{n^2}=1 \Rightarrow x=\frac{n}{n-1}$.\\
Hence, an unbiased estimator can be $\frac{n}{n-1}\bar{X_i}(1-\bar{X_i})$





\end{problem}

\begin{problem}{2}
\item 1.
$(\mathbb{N}, (Poiss(\lambda))_{\lambda>0})$. This paramter is identified.
\item 2.
$(\mathbb{R_+}, (Exp(\lambda))_{10>\lambda>0})$. This parameter is identified.
\item 3.
$(\mathbb{R_+}, (Uni(0, \theta))_{\theta >0})$. This parameter is identified.
\item 4.
$(\mathbb{R}, (N(\mu, \sigma^2))_{(\mu, \sigma^2) \in \mathbb{R} \times \mathbb{R_+}})$. These parater are identified.
\item 5.
\[\mathbb{P}(N(\mu, \sigma^2)>0)=\mathbb{P}\Big( N(0,1) > \frac{-\mu}{\sigma^2} \Big)=\phi(\frac{\mu}{\sigma^2})\]
Hence, the statistical model is: $(\{0,1\}, (Ber(\phi(\frac{\mu}{\sigma^2}))_{(\mu, \sigma^2) \in \mathbb{R} \times \mathbb{R_+}})$. This model depends on $\frac{\mu}{\sigma^2} \Rightarrow$ these parameters are not identified.
\item 6.
Same for 3.
\item 7.
Let X $\thicksim Exp(\lambda) \Rightarrow \mathbb{P}(X>20)=e^{-20\lambda}$. Hence, the statistical model is:
\[(\{ 0,1\},(Ber(e^{-20\lambda}))_{\lambda>0}) \] 
This parameter is identified.

\item 8.
Let X $\thicksim Ber(p)$ such that:
\begin{align}
    \begin{cases}
        X_{i}=1 \text{ if machine i has timelife less than 500 days} \\
        X_{i}=0 \text{ otherwise}
    \end{cases}
\end{align}

Hence: 
\[p=\mathbb{P}(X_{i}=1)=1-e^{-500\lambda}\]
The number of machines that have stopped working before 500 days is a binominal random variable with parameter (67, $1-e^{-500\lambda}$)\\
The statistical model is $(\{1,2,3,..,67 \}, (Binominal(67, 1-e^{-500\lambda}))_{\lambda>0})$. This parameter is identified.

\end{problem}

\begin{problem}{3}
\item 1.
By central limit theorem (CLT), we have: 
\[ \frac{\sqrt{n}(\bar{X_i}-\mu)}{\sigma} \thicksim \text( N(0,1))\]
Hence, $(a_{n})_{n \in \mathbb{N}}$ can be $\frac{\sqrt{n}}{\sigma}$ and $(b_{n})_{n \in \mathbb{N}}$ can be $\mu$.

\item 2.
We have: $Z \backsim N(0,1)$\\
Hence, $\mathbb{P}[|Z| \leqslant t]=\mathbb{P}[-t\leqslant Z \leqslant t]= \phi(t)-\phi(-t)= \phi(t)-(1-\phi(t))=2\phi(t)-1= 2\mathbb{P}[Z \leqslant t]-1$.
\item 3.
From part 1 we get: 
\[ \frac{\sqrt{n}(\bar{X_i}-\mu)}{\sigma} \thicksim \text( N(0,1))\]
from part 2 we get:
\[ \mathbb{P}[|Z| \leqslant t]=2\mathbb{P}[Z \leqslant t]-1\]
Substitution:
\[ \mathbb{P}\Big[|\frac{\sqrt{n}(\bar{X_i}-\mu)}{\sigma}| \leqslant t\Big]=2\mathbb{P}[Z \leqslant t]-1\]
We have $2\mathbb{P}[Z \leqslant t]-1=0.95 \Rightarrow t=\phi^{-1}(\frac{0.95+1}{2})=1.96$. \\
Hence, 
\[ \mathbb{P}\Big[|\frac{\sqrt{n}(\bar{X_i}-\mu)}{\sigma}| \leqslant 1.96\Big]= 0.95\]

\[  \mathbb{P}\Big[-\frac{\sqrt{n}(\bar{X_i}-\mu)}{\sigma} \leqslant 1.96 \leqslant \frac{\sqrt{n}(\bar{X_i}-\mu)}{\sigma}\Big]= 0.95\]
Because $X_i$ is Poisson random variable with parameter $\lambda$, so $ \mu=\lambda$ and $\sigma= \sqrt{\lambda}$\\
We get:
\[  \mathbb{P}\Big[-\frac{\sqrt{n}(\bar{X_i}-\lambda)}{\sqrt{\lambda}} \leqslant 1.96 \leqslant \frac{\sqrt{n}(\bar{X_i}-\lambda)}{\sqrt{\lambda}}\Big]= 0.95\]

\[ \mathbb{P}\Big[\bar{X_i}- \frac{1.96\sqrt{\lambda}}{\sqrt{n}} \leqslant \lambda \leqslant \bar{X_i}+ \frac{1.96\sqrt{\lambda}}{\sqrt{n}}  \Big] =0.95\]

We know: $\bar{X_i} \xrightarrow{P} \mathbb{E}[\bar{X_i}]=\lambda$\\
Hence, 
\[ \mathbb{P}\Big[\bar{X_i}- \frac{1.96\sqrt{\bar{X_i}}}{\sqrt{n}} \leqslant \lambda \leqslant \bar{X_i}+ \frac{1.96\sqrt{\bar{X_i}}}{\sqrt{n}}  \Big] \geqslant 0.95\]
$\Rightarrow$ L=[$\bar{X_i}- \frac{1.96\sqrt{\bar{X_i}}}{\sqrt{n}}, \bar{X_i}+ \frac{1.96\sqrt{\bar{X_i}}}{\sqrt{n}}$] 
\item 4.
We can easily see min($X_i$)$\leqslant \bar{X_i} \leqslant$ max($X_i$). Hence, a new interval can be: 
\[L=[min(X_i)- \frac{1.96\sqrt{\bar{X_i}}}{\sqrt{n}}, max(X_i)+ \frac{1.96\sqrt{\bar{X_i}}}{\sqrt{n}}]\]
\end{problem}
\begin{problem}{4}
We have $X_i$ is IID. Hence, $\mathbb{P}(M_n \leqslant t)=\prod_{n=1}^{n}\mathbb{P}(X_i\leqslant t).$\\
By uniform distribution, the CDF of $M_n$:
\[ \mathbb{P}(M_n \leqslant t)=F(t)=\Big(\frac{t}{\theta}\Big)^n \]
Hence, the PDF of $M_n$ is:
\[f(t)=\frac{dF}{dt}=n\theta^{-n}t^{n-1} \]
We can easily get:
\[\mathbb{E}[M_n]= \int_{0}^{\theta} tn\theta^{-n}t^{n-1}dt=\frac{n}{n+1}\theta \rightarrow \theta \textbf{ as n}\rightarrow \infty \]
By Markov's Inequality:
\[ \mathbb{P}\Big[ | M_n -\theta|> \epsilon \Big]\leqslant \mathbb{P}[M_n-\theta>\epsilon] \leqslant \frac{\mathbb{E}[M_n-\theta]}{\epsilon}=\frac{\mathbb{E}[M_n]-\theta}{\epsilon} \rightarrow 0\]
Hence, $M_n$ converages in probility to $\theta$.

\item 2.
From part 1 we get: $M_n$: $\mathbb{P}[M_n \leqslant t]=\Big(\frac{t}{\theta}\Big)^n$. Hence, CDF of $n(1-\frac{M_n}{\theta})$ is:
\[ P\Big[n(1-\frac{M_n}{\theta})\leqslant t\Big]=\mathbb{P}\Big[M_n \geqslant \frac{(n-t)\theta}{n} \Big] =1-\Big(\frac{n-t}{n}\Big)^{n} \rightarrow 1- e^{-t} \textbf{ as n}  \rightarrow \infty\]

Hence, $n(1-\frac{M_n}{\theta})$ converages in distribution to an exponential random variable with parameter 1.
\item 3.
Let A is an exponential random variable with parameter 1. Because $n(1-\frac{M_n}{\theta})$ converages in distribution to X, we have:
\[\mathbb{P}\Big[ n(1-\frac{M_n}{\theta}) \leqslant t\Big] \rightarrow \mathbb{P}[X\leqslant t]= 1-e^{-t} \]
$1-e^{-t}=0.95 \Rightarrow t=3$. We have:
\[ \mathbb{P}\Big[ n(1-\frac{M_n}{\theta}) \leqslant 3\Big] \rightarrow 0.95\]
which is:
\[ \mathbb{P}\Big[ \theta \leqslant \frac{nM_n}{n-3}\Big] \rightarrow 0.95\]
On the other hand, we always have $\theta \geqslant M_n$ (uniform distribution). Hence, we get: 
\[ \mathbb{P}\Big[M_n \leqslant \theta \leqslant \frac{nM_n}{n-3}\Big] \rightarrow 0.95 \textbf{ as n} \rightarrow \infty\]
We conclude $L=\Big[M_n, \frac{nM_n}{n-3} \Big]=\Big[ M_n, M_n+ \frac{3M_n}{n-3} \Big]$.
\item 4.
$bias(M_n)=\mathbb{M_n}-\theta= \frac{n}{n+1}\theta -\theta \neq 0$. Hence, $M_n$ is biased.

As described in the doc, $\boldsymbol{y}$ is a one-hot vector with a 1 for the true outside word $o$, that means $y_i$ is 1 if and only if $i == o$. so the proof could be below:
<!-- $ - \sum_{w\in Vocab}y_w\log(\hat{y}_o) = $ -->

$\begin{aligned}
    - \sum_{w\in Vocab}y_w\log(\hat{y}_w) &= - [y_1\log(\hat{y}_1) + \cdots + y_o\log(\hat{y}_o) + \cdots + y_w\log(\hat{y}_w)] \\
    & = - y_o\log(\hat{y}_o) \\
    & = -\log(\hat{y}_o) \\
    & = -\log \mathrm{P}(O = o | C = c)
\end{aligned}$

**(b)** we know this deravatives:
$$
\because J = CE(y, \hat{y}) \\
\hat{y} = softmax(\theta)\\
\therefore \frac{\partial J}{\partial \theta} = (\hat{y} - y)^T
$$

$y$ is a column vector in the above equation. So, we can use chain rules to solve the deravitive:

$$\begin{aligned}
\frac{\partial J}{\partial v_c} &= \frac{\partial J}{\partial \theta} \frac{\partial \theta}{\partial v_c} \\
&= (\hat{y} - y) \frac{\partial U^Tv_c}{\partial v_c} \\
&= U^T(\hat{y} - y)^T
\end{aligned}$$

**(c)**
similar to the equation above.
$$\begin{aligned}
\frac{\partial J}{\partial v_c} &= \frac{\partial J}{\partial \theta} \frac{\partial \theta}{\partial U} \\
&= (\hat{y} - y) \frac{\partial U^Tv_c}{\partial U} \\
&= v_c(\hat{y} - y)^T
\end{aligned}$$
\end{problem}
\end{document}