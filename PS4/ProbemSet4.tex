\documentclass[10pt]{article}
\usepackage{hyperref}
\usepackage[margin=1in]{geometry} 
\usepackage{amsmath,amsthm,amssymb, graphicx, multicol, array}
\usepackage{bbold}
\newcommand{\N}{\mathbb{N}}
\newcommand{\Z}{\mathbb{Z}}
 
\newenvironment{problem}[2][Problem]{\begin{trivlist}
\item[\hskip \labelsep {\bfseries #1}\hskip \labelsep {\bfseries #2.}]}{\end{trivlist}}



\begin{document}
\title{Problem Set 4}
\author{Hoang Nguyen, Huy Nguyen}
\maketitle
    
\begin{problem}{1}
\item 1.
We have the log-likelihood function:  
\[\ell_n(p) = log(\prod_{i=1}^{n}p^{X_i}(1-p)^{1-X_i})=log(p)\sum_{i=1}^{n}X_i + log(1-p)\sum_{i=1}^{n}(1-X_i) \]
Hence,
\[\frac{\partial\ell_n(p)}{\partial p}=\frac{\sum_{i=1}^{n}X_i}{p} - \frac{\sum_{i=1}^{n}(1 - X_i)}{1-p} \]
Set this equals to 0, we get: $p=\frac{1}{n}\sum_{i=1}^{n}X_i$.\\
We have:
\[I(\theta) =\mathbb{E}[(\frac{\partial\ell_n(p)}{\partial p})^2]=\mathbb{E}\Big[(\frac{X}{p} - \frac{1-X}{1-p})^2 \Big] = \mathbb{E}[\frac{X^2}{p^2}] - 2\mathbb{E}[\frac{X - X^2}{p(1-p)}] + \mathbb{E}[\frac{X^2-2X+1}{(1-p)^2}] \]
From Bernoulli distribution, we know: $\mathbb{E}[X]=p$ and $\mathbb{E}[X^2]=p$. Plugging those thing to equation above, we get:
\[ I(\theta)=\frac{p}{p^2} - 2\frac{0-0}{p(1-p)} + \frac{p-2p+1}{(1-p)^2}=\frac{1}{p(1-p)} \]

\item 2. 
We have the log-likelihood function:  
\[\ell_n(\lambda) = log(\prod_{i=1}^{n}e^{-\lambda} \frac{\lambda^{X_i}}{X_i!})=-n\lambda -\sum_{i=1}^{n}log(X_i!) + log(\lambda)\sum_{i=1}^{n}X_i \]
Hence,
\[\frac{\partial\ell_n(\lambda)}{\partial \lambda}= -n + \frac{1}{\lambda}\sum_{i=1}^{n}X_i \]
Set this equals to 0, we get: $\lambda=\frac{1}{n}\sum_{i=1}^{n}X_i$.\\
We have: $\frac{\partial^2\ell_n(\lambda)}{\partial \lambda^2}= \frac{\partial (-1 + \frac{X}{\lambda})}{\partial \lambda}=\frac{-X}{\lambda^2}$. Hence, $I(\lambda)= -\mathbb{E}[\frac{\partial^2\ell_n(\lambda)}{\partial \lambda^2}]= -\mathbb{E}[-\frac{X}{\lambda^2}]=\frac{1}{\lambda}$\\



\item 3. 
We have the log-likelihood function:  
\[\ell_n(\lambda) = log(\lambda^n e^{-\lambda \sum_{i=1}^{n} X})=nlog(\lambda)-\lambda\sum_{i=1}^{n}X_i \]
Hence,
\[\frac{\partial\ell_n(\lambda)}{\partial \lambda}=\frac{n}{\lambda} - \sum_{i=1}^{n}X_i \]
Set this equals to 0, we get: $\lambda=\frac{n}{\sum_{i=1}^{n}X_i}$.\\
We have: $\frac{\partial^2\ell_n(\lambda)}{\partial \lambda^2}= \frac{\partial (\frac{1}{\lambda} - X)}{\partial \lambda}=\frac{-1}{\lambda^2}$. Hence, $I(\lambda)= -\mathbb{E}[\frac{\partial^2\ell_n(\lambda)}{\partial \lambda^2}]= -\mathbb{E}[\frac{-1}{\lambda^2}]=\frac{1}{\lambda^2}$\\
\item 4.
We have this result from problem 2 in Problem 3: \[\begin{cases} \mu=\bar{X_n}\\ \sigma^2=\frac{\sum_{i=1}^n(X_i-\bar{X_n})^2}{n}\end{cases}\]
Recall from problem 2 in Problem 3:
\[ \begin{cases} \frac{\partial \ell_n(\mu, \sigma^2)}{\partial \mu} = \frac{1}{\sigma^2}(X - \mu) \\\frac{\partial \ell_n(\mu, \sigma^2)}{\partial \sigma^2} = \frac{-1}{2\sigma^2} + \frac{(x-\mu)^2}{2\sigma^4} \end{cases} \Leftrightarrow \begin{cases} \frac{\partial^2  \ell_n(\mu, \sigma^2)}{\partial \mu^2} = \frac{-1}{\sigma^2} \\ \frac{\partial^2  \ell_n(\mu, \sigma^2)}{\partial \mu \sigma^2}= \frac{\partial^2  \ell_n(\mu, \sigma^2)}{\partial \sigma^2 \mu}= \frac{-2(X-\mu)}{\sigma^3} \\  \frac{\partial^2  \ell_n(\mu, \sigma^2)}{\partial \sigma^2 \sigma^2} = \frac{1}{2\sigma^4} - \frac{(X-\mu)^2}{2\sigma^6} \end{cases}\]
Hence, 
\[ I(\mu, \sigma^2)=\mathbb{E} \begin{bmatrix}
    \frac{\partial \ell_n(\mu, \sigma^2)}{\partial \mu^2} & \frac{\partial \ell_n(\mu, \sigma^2)}{\partial \mu \sigma^2}\\
    \frac{\partial \ell_n(\mu, \sigma^2)}{\partial \sigma^2 \mu} & \frac{\partial \ell_n(\mu, \sigma^2)}{\partial \sigma^2 \sigma^2}
\end{bmatrix} = -\mathbb{E} \begin{bmatrix}
    -\frac{1}{\sigma^2} & \frac{-2(X-\mu)}{\sigma^3} \\
    \frac{-2(X-\mu)}{\sigma^3} & \frac{1}{2\sigma^4} - \frac{(X-\mu)^2}{2\sigma^6}
\end{bmatrix}= \begin{bmatrix}
\frac{1}{\sigma^2} & 0 \\
0 & \frac{1}{2\sigma^4}
\end{bmatrix} \]
\item 5.
We have the log-likelihood function:  
\[\ell_n(\lambda, a) = log(\prod_{i=1}^{n}\lambda e^{-\lambda(X_i-a)} \mathbb{1}_{x \geq a}) = \sum_{i=1}^{n}= nlog(\lambda) - \lambda \sum_{i=1}^{n}(X_i-a) +\sum_{i=1}^{n}log(\mathbb{1}_{X_i \geq a}) \]
Hence,
\[\frac{\partial\ell_n(\lambda, a)}{\partial a}=  n\lambda   \]
Because $n\lambda $ for all $\lambda > 0$, $\ell_n(\lambda, a)$ is an increasing function of $a$ until $a > min(X_i)$. Hence $\ell_n(\lambda, a)$ is maximal with respect to $a$ when $a$ is made as large as possible without exceeding the minimum order statistic $\Longrightarrow a=min(X_i)$\\
Let's denote $y=x-a$, hence $f_{y}(y)=\frac{d}{dx}(x-a)f_{x}{y}=f_{x}(y)$ which is exponential distribution.\\ 
From question 3, we know $\lambda=\frac{n}{\sum_{i=1}^{n}Y_i}=\frac{n}{\sum_{i=1}^{n}(X_i-a)}$\\
We have: 
\[ \begin{cases} \frac{\partial\ell_n(\lambda, a)}{\partial \lambda} =   \frac{1}{\lambda} - (X -a) \\ \frac{\partial\ell_n(\lambda, a)}{\partial a} = \lambda   \end{cases} \Leftrightarrow \begin{cases}  \frac{\partial^2\ell_n(\lambda, a)}{\partial \lambda^2} = \frac{-1}{\lambda^2} \\ \frac{\partial^2\ell_n(\lambda, a)}{\partial \lambda a} = \frac{\partial^2\ell_n(\lambda, a)}{\partial a\lambda}= 1 \\ \frac{\partial^2\ell_n(\lambda, a)}{\partial a^2} = 0 \end{cases}\]
Hence,
\[ I(\lambda , a)=\mathbb{E} \begin{bmatrix}
    \frac{\partial \ell_n(\lambda, a)}{\partial \lambda^2} & \frac{\partial \ell_n(\lambda, a)}{\partial \lambda a}\\
    \frac{\partial \ell_n(\lambda, a)}{\partial a \lambda} & \frac{\partial \ell_n(\lambda, a)}{\partial a^2}
\end{bmatrix} = -\mathbb{E} \begin{bmatrix}
    -\frac{1}{\lambda^2} & 1 \\
    1 & 0
\end{bmatrix}= \begin{bmatrix}
\frac{1}{\lambda^2} & 1 \\
1 & 0
\end{bmatrix} \]


\item 6.
We have the log-likelihood function:  
\[\ell_n(\mu, \sigma^2) = log(\prod_{i=1}^{n}\frac{1}{X_{i}\sqrt{2\pi\sigma^2}}e^{-\frac{1}{2\sigma^2} (log(X_i) - \mu)^2})= -\frac{n}{2}log(2\pi) - \frac{n}{2}log(\sigma^2) -\sum_{i=1}^{n}log(X_{i})  -\frac{1}{2\sigma^2}\sum_{i=1}^{n}(log(X_i)-\mu)^2 \]
Hence,
\[\frac{\partial\ell_n(\mu, \sigma^2)}{\partial \mu}=   \frac{1}{\sigma^2}\Big ( \sum_{i=1}^{n}log(X_i) -n\mu\Big )\]
Set this equals to 0, we get: $\mu=\frac{1}{n}\sum_{i=1}^{n}log(X_i)$.\\
On the other hand,
\[\frac{\partial\ell_n(\mu, \sigma^2)}{\partial \sigma^2} = \frac{-n}{2\sigma^2}+ \frac{\sum_{i=1}^{n} (log(X_i) - \mu)^2}{2\sigma^4} \]
Set this equals to 0, we get: $\sigma^2=\frac{1}{n}\sum_{i=1}^{n}(log(X_i)-\mu)^2$.\\


We have: 
\[ \begin{cases} \frac{\partial \ell_n(\mu, \sigma^2)}{\partial \mu} = \frac{1}{\sigma^2}(log(X) - \mu) \\\frac{\partial \ell_n(\mu, \sigma^2)}{\partial \sigma^2} = \frac{-1}{2\sigma^2} + \frac{(log(x)-\mu)^2}{2\sigma^4} \end{cases} \Leftrightarrow \begin{cases} \frac{\partial^2  \ell_n(\mu, \sigma^2)}{\partial \mu^2} = \frac{-1}{\sigma^2} \\ \frac{\partial^2  \ell_n(\mu, \sigma^2)}{\partial \mu \sigma^2}= \frac{\partial^2  \ell_n(\mu, \sigma^2)}{\partial \sigma^2 \mu}= \frac{-2(log(X)-\mu)}{\sigma^3} \\  \frac{\partial^2  \ell_n(\mu, \sigma^2)}{\partial \sigma^2 \sigma^2} = \frac{1}{2\sigma^4} - \frac{(log(X)-\mu)^2}{2\sigma^6} \end{cases}\]
Hence, 
\[ I(\mu, \sigma^2)=\mathbb{E} \begin{bmatrix}
    \frac{\partial \ell_n(\mu, \sigma^2)}{\partial \mu^2} & \frac{\partial \ell_n(\mu, \sigma^2)}{\partial \mu \sigma^2}\\
    \frac{\partial \ell_n(\mu, \sigma^2)}{\partial \sigma^2 \mu} & \frac{\partial \ell_n(\mu, \sigma^2)}{\partial \sigma^2 \sigma^2}
\end{bmatrix} = -\mathbb{E} \begin{bmatrix}
    -\frac{1}{\sigma^2} & \frac{-2(log(X)-\mu)}{\sigma^3} \\
    \frac{-2(log(X)-\mu)}{\sigma^3} & \frac{1}{2\sigma^4} - \frac{(log(X)-\mu)^2}{2\sigma^6}
\end{bmatrix}= \begin{bmatrix}
\frac{1}{\sigma^2} & 0 \\
0 & \frac{1}{2\sigma^4}
\end{bmatrix} \]
The last equation comes from the fact that $ln(X) \sim N(\mu, \sigma^2)$ because of the following reason:\\
Let $y=log(x)$ 
\[ f_{y}(y)=f_{x}(e^y)e^{y}=\frac{e^{y}}{e^{y}\sqrt{2\pi \sigma^2}} e^{-\frac{1}{2\sigma^2}(log(e^y)-\mu)^2}=\frac{1}{\sqrt{2\pi \sigma^2}}e^{-\frac{1}{2\sigma^2}(y-\mu)^2} \]
Hence, $y \sim N(\mu, \sigma^2) \Longrightarrow \mathbb{E}[y]=\mathbb{E}[log(x)]=\mu \Longrightarrow \mathbb{E}[\log(x) -\mu] = 0$.
There is an another better approach for the first part of this question (maximum likelihood). Let $y=log(x)$, we have $y\sim N(\mu, \sigma^2)$ (which was proved a few lines above). Now, We come up with question 4.\\












\end{problem}

\begin{problem}{2}
\item 1. We have the first moment $m_{1}=\mathbb{E}[X]=p$. Hence, $p_{MM}=\frac{1}{n}\sum_{i=1}^{n}X_i$.\\
\item 2. We have the first moment $m_{1}=\mathbb{E}[X]=\lambda$. Hence, $\lambda_{MM}=\frac{1}{n}\sum_{i=1}^{n}X_i$.\\
\item 3. We have the first moment $m_{1}=\mathbb{E}[X]=\frac{1}{\lambda}$. Hence, $\frac{1}{\lambda_{MM}}=\frac{1}{n}\sum_{i=1}^{n} \Leftrightarrow \lambda_{MM}=\frac{n}{\sum_{i=1}^{n}X_i}$.\\
\item 4. We have:
\[ \begin{cases} m_{1}=\mathbb{E}[X]=\mu \\ m_{2}=\mathbb{E}[X^2]=\mu^2 + \sigma^2 \end{cases}\]
Hence, 
\[ \begin{cases} \mu_{MM}=\frac{1}{n}\sum_{i=1}^{n}X_i \\ \mu_{MM}^2 + \sigma_{MM}^2 = \frac{1}{n}\sum_{i=1}^{n}X_{i}^{2} \end{cases} \Leftrightarrow \begin{cases} \mu_{MM}=\frac{1}{n}\sum_{i=1}^{n}X_i \\ \sigma_{MM}^2 = \frac{1}{n}\sum_{i=1}^{n}X_{i}^{2} - \mu_{MM}^2 \end{cases} \]
\item 5. 
Let's denote $y=x-a$, from question 5 problem 1 we have $y \sim Exp(\lambda)$. Hence: 
\[ \begin{cases} \mathbb{E}[X]=\mathbb{E}[Y]+a=\frac{1}{\lambda} + a\\
\mathbb{E}[X^2]=\mathbb{E}[(Y+a)^2]= \frac{2}{\lambda^2} + \frac{2a}{\lambda} + a^2 \end{cases}\].
We have system equation:
\[ \begin{cases} \frac{1}{\lambda_{MM}} + a_{MM}= \bar{X_n}\\
 \frac{2}{\lambda_{MM}^2} + \frac{2a_{MM}}{\lambda_{MM}} + a_{MM}^2 = \bar{X_n}^2  \end{cases}  \Leftrightarrow  \begin{cases} \lambda_{MM} = \frac{1}{\sqrt{\bar{X_n}^2- \bar{X_n}}}\\ a = \bar{X_n} - \frac{1}{\lambda_{MM}} \end{cases}\].
 
\item 6.
Let's denote $Y=log(X)$. From question 6 Problem 1 we know $Y \sim N(\mu, \sigma)$\\
Form question 4, we have 
\[  \begin{cases} \mu_{MM}=\frac{1}{n}\sum_{i=1}^{n}Y_i = \frac{1}{n}\sum_{i=1}^{n}log(X_i) \\ \sigma_{MM}^2 = \frac{1}{n}\sum_{i=1}^{n}Y_{i}^{2} - \mu_{MM}^2 = \frac{1}{n}\sum_{i=1}^{n}log(X_{i})^{2} - \mu_{MM}^2 \end{cases} \] 



\end{problem}
\begin{problem}{3}
\item 1. $\mathbb{P}[X=1]=\mathbb{P}[Exp(\lambda) > z]= e^{-\lambda z} $. Hence $X_{i}$'s follow Bernoulli distribution with parameter $p=e^{-\lambda z}$.
\item 2. We have $\bar{X_n}=\frac{1}{n}\sum_{i=1}^{n}X_i$. By CLL, $\bar{X_n} \sim N(\mathbb{E}[X], \frac{\mathbb{V}[X]}{n})$. From question 1, we have $\mathbb{E}[X]=p=e^{-\lambda z}$ and $\mathbb{V}[X]=p(1-p)=e^{-\lambda z}(1- e^{-\lambda z})$. Hence, $\bar{X_n} \sim N(e^{-\lambda z}, \frac{e^{-\lambda z}(1- e^{-\lambda z})}{n} )$.
\item 3. From question 1 problem 1 we have $\bar{X_n}$ is a consistent estimator of $p =  e^{-\lambda z}$. Hence, $\frac{-log(\bar{X_n})}{z}$ is a consistent estimator of $\lambda$. Hence, $f(\bar{X_n})=\frac{-log(\bar{X_n})}{z}$


\item 4. By CLT, we have $\sqrt{n}(\bar{X_n} -p) \longrightarrow N(0, p(1-p))$\\
On the other hand, let $f(x)=\frac{-log(x)}{z} \Rightarrow \frac{\partial f(x)}{\partial x}=\frac{-1}{zx} \Rightarrow \frac{\partial f(x)}{\partial x} (p)=\frac{-1}{zp}$.
Using Delta-method, we know:
\[\sqrt{n}(f(\bar{X_n}) - f(p))  \longrightarrow N(0, (\frac{-1}{zp})^2 p(1-p)) \Leftrightarrow \sqrt{n}(f(\bar{X_n}) - \frac{-log(p)}{z})  \longrightarrow N(0, \frac{1-p}{z^2p})\]

\item 5.  
Let denote $f(p)= \frac{1-p}{z^2p}$. Hence, $\frac{\partial f(p)}{\partial p}= \frac{-1}{z^2p^2} \leqslant 0  \Rightarrow argmin f(p)=1$. Hence, $e^{-\lambda z}=1 \Leftrightarrow g_{\lambda}(z)= ze^{-\lambda z}=z $
\item 6.\\ 
a) The statistical model is ($\mathbb{R}+$, ($Exp(\lambda))_{\lambda>0}$).\\
b) From question 3 problem 1, we know $I_{Y}(\lambda)=\frac{1}{\lambda^2}$.\\
c) From question 1 problem 1, we know $I_{X}(\lambda)=\frac{1}{p(1-p)}=\frac{1}{e^{-\lambda z}(1-e^{-\lambda z})}$.\\
d) I wonder what value of threshold z is?

\end{problem}

\end{document}